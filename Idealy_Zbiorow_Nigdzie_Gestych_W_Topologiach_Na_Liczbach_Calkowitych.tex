\newcommand{\real}{\mathbb{R}}
\newcommand{\nnatural}{\mathbb{N}}
\newcommand{\cE}{\mathcal{E}}
\newcommand{\cF}{\mathcal{F}}
\newcommand{\cI}{\mathcal{I}}
\newcommand{\cM}{{\mathcal{M}}}
\newcommand{\cN}{{\mathcal{N}}}
\newcommand{\cIMN}{\mathcal{I}_{\mathcal{N}, \mathcal{M}}}
\newcommand{\Fix}{\mathit{Fix}}
\newcommand{\B}{\mathcal{B}}
\newcommand{\SqrFr}{\mathbb{SF}}

\documentclass{beamer} 
%\documentclass[a4paper, 11pt, xcolor=dvipsnames]{beamer} 
\usepackage[polish]{babel}
\usepackage[utf8]{inputenc}
\usepackage{t1enc}
\usepackage{amsmath}
\usepackage{graphics} 
\usepackage{fancybox} 

%%%\newcommand{\arithseq}[2]{\{#1n + #2\}}
\newcommand{\arithseq}[2]{\langle#2, #1\rangle}
\newcommand{\Arithseq}[2]{\Big\langle#2, #1\Big\rangle}
\newcommand{\T}{\mathcal{T}}

%\usepackage{helvet}
%All font families are not available in every Beamer installation, but
%typically, at least some of the following families will be available:
%    serif          avant        bookman      chancery        charter
%     euler         helvet       mathtime       mathptm       mathptmx
%    newcent      palatino        pifont        utopia

\mode<presentation> {
\usetheme{Madrid} 
}
\usepackage{graphicx} % Allows including images
\usepackage{booktabs} % Allows the use of \toprule, \midrule and \bottomrule in tables

\usecolortheme[rgb={0.1,0.6,0.3}]{structure} 
%\usecolortheme[named=RubineRed]{structure} 
%\usecolortheme[named=CarnationPink]{structure}
%\usecolortheme[rgb={0.1,0.6,0.3}]{structure} 
\begin{document} 
%\setbeamercovered{transparent=15} %Ukryte (np przez uncover beda polwidoczne)
\setbeamercovered{dynamic}        %Ukryte (np przez uncover beda polwidoczne, im blizej tym mocniej)
%%% to ponizej tez nie dziala...
%\transglitter[direction=90]
%%% to nizej mialo zmienic kolor bibliografii lecz niestety nie zadzialalo.
%\setbeamercolor{bibliography item}{fg=black}
%\setbeamercolor*{bibliography entry title}{fg=black}
%-----------------------------------
\title[On special kind of ideals...]{
On special kind of ideals of the real line generated by partitions into measure null and first category sets.
} 
%\subtitle{}
\author{Marta Kwela \& AN} 
\institute[IM UG]{
  Universytet Gdański \\
  Institut Matematyki \\
  Wita Stwosza 57 \\
  80 -- 952 Gdańsk \\[1ex]
  \textit{e-mail: andrzej@mat.ug.edu.pl} 
}
\date{Bydgoszcz \today}
%---------------------
\begin{frame} 
  \titlepage 
\end{frame} 
%---------------------
%\section[Outline of the talk]{}
%\frame{\tableofcontents}
%----------------------------------
%\section{Introduction}
\begin{frame}
\uncover<1->{
\begin{block}{Niezbędne definicje}
Oznaczenie: $\arithseq{a}{b} = \{an + b\colon n\in \cN_0\}$

Rozważamy następujące topologie na $\cN$:

\begin{itemize}
\item \emph{Topologia Furstenberga} $\T_F$ \\
z bazą: $\B_F = \{\arithseq{a}{b} :\ b\leq a\}$,
\item \emph{Topologia Golomba} $\T_G$ \\
z bazą: $\B_G = \{\arithseq{a}{b} :\ (a,b)=1,\ b<a\}$,
\item \emph{Topologia Kircha} $\T_K$ \\
with the base $\B_K = \{\arithseq{a}{b} :\ (a,b)=1,\ b<a,\ a\in\SqrFr\}$.
\end{itemize}



\end{block}
}

\uncover<2->{
\begin{block}{Definition} 
  \begin{itemize}
  \item<2->
(...)
  \end{itemize}
\end{block}
}
\uncover<3->{
\begin{block}{Example}
(...)
\end{block}
}
\end{frame}
%---------------------------------------------
%---------------------------------------------
\begin{frame}{allowframebreaks}
\frametitle{This talk is based on}
\beamertemplatebookbibitems
\begin{thebibliography}{10}{
\bibitem{MZ}
  {\sc M.Michalski, Sz.Żeberski}, 
  {\em Some properties of ${\mathcal I}-Luzin sets,$}
  Topology and its Applications 55 (2015) Vol 198, 122-135.
}
\end{thebibliography}
\end{frame}

\begin{frame}\frametitle{The last slide}
%\bigskip
\begin{center}{\Huge Thank You}\end{center}
\begin{center}{\Huge for}\end{center}
\begin{center}{\Huge Your Attention}\end{center}
%%%\begin{center}\includegraphics[height=2cm]{smile.png}\end{center}
\end{frame}

\end{document}

