\newcommand{\real}{\mathbb{R}}
\newcommand{\nnatural}{\mathbb{N}}
\newcommand{\cE}{\mathcal{E}}
\newcommand{\cF}{\mathcal{F}}
\newcommand{\cI}{\mathcal{I}}
\newcommand{\cM}{{\mathcal{M}}}
\newcommand{\cN}{{\mathcal{N}}}
\newcommand{\cIMN}{\mathcal{I}_{\mathcal{N}, \mathcal{M}}}
\newcommand{\Fix}{\mathit{Fix}}

\newcommand{\N}{{\mathbb N}}
\newcommand{\Z}{{\mathbb Z}}
\newcommand{\R}{{\mathbb R}}
\newcommand{\Q}{{\mathbb Q}}
\newcommand{\Fin}{\textrm{Fin}}
\newcommand{\I}{\mathcal I}
\newcommand{\J}{\mathcal J}
\newcommand{\T}{\mathcal{T}}
\newcommand{\B}{\mathcal{B}}
\newcommand{\SqrFr}{\mathbb{SF}}
\newcommand{\calF}{\mathcal{F}}
\newcommand{\modulo}{\textrm{mod }}
\newcommand{\InfSubs}{[\N]^{\omega}}
\newcommand{\finbw}{\text{FinBW}}
\newcommand{\FinPart}{(\N)^{< \omega}}
\newcommand{\MB}{S^0}
\newcommand{\MBC}{\mathcal{MBC}}
\newcommand{\Seg}{\mathrm{Seg}}
\newcommand{\NULL}{\mathrm{NULL}}
\newcommand{\NWD}{\mathrm{NWD}}
\newcommand{\INULL}{\I_\mathrm{NULL}}			  
\newcommand{\cl}{\mathrm{cl}}
\newcommand{\interior}{\mathrm{int}}
\documentclass{beamer} 
%\documentclass[a4paper, 11pt, xcolor=dvipsnames]{beamer} 
\usepackage[polish]{babel}
\usepackage[utf8]{inputenc}
\usepackage{t1enc}
\usepackage{amsmath}
\usepackage{graphics} 
\usepackage{fancybox} 

%%%\newcommand{\arithseq}[2]{\{#1n + #2\}}
\newcommand{\arithseq}[2]{\langle#2, #1\rangle}
\newcommand{\Arithseq}[2]{\Big\langle#2, #1\Big\rangle}
%%% macros for the picture environment
\newcommand{\putrectangle}[4]{
  \multiput(#1,#2)(#3,0){2}{\line(0,1){#4}}
  \multiput(#1,#2)(0,#4){2}{\line(1,0){#3}}
}

%\usepackage{helvet}
%All font families are not available in every Beamer installation, but
%typically, at least some of the following families will be available:
%    serif          avant        bookman      chancery        charter
%     euler         helvet       mathtime       mathptm       mathptmx
%    newcent      palatino        pifont        utopia

\mode<presentation> {
\usetheme{Madrid} 
}
\usepackage{graphicx} % Allows including images
\usepackage{booktabs} % Allows the use of \toprule, \midrule and \bottomrule in tables
%%%%% pakiety spoza prezentacji - ,,spadly'' z macierzystego pliku
\usepackage{amssymb}
\usepackage{amsmath}
\usepackage{latexsym}

\usecolortheme[rgb={0.1,0.6,0.3}]{structure} 
%\usecolortheme[named=RubineRed]{structure} 
%\usecolortheme[named=CarnationPink]{structure}
%\usecolortheme[rgb={0.1,0.6,0.3}]{structure} 
\begin{document} 
%\setbeamercovered{transparent=15} %Ukryte (np przez uncover beda polwidoczne)
%\setbeamercovered{dynamic}        %Ukryte (np przez uncover beda polwidoczne, im blizej tym mocniej)
\setbeamercovered{transparent=0}
%%% to ponizej tez nie dziala...
%\transglitter[direction=90]
%%% to nizej mialo zmienic kolor bibliografii lecz niestety nie zadzialalo.
%\setbeamercolor{bibliography item}{fg=black}
%\setbeamercolor*{bibliography entry title}{fg=black}
%-----------------------------------
\title[O ideałach zbiorów ND...]{
O ideałach zbiorów nigdziegęstych w topologiach
na zbiorze liczb naturalnych.
} 
%\subtitle{}
\author{Marta Kwela \& AN} 
\institute[IM UG]{
  Universytet Gdański \\
  Institut Matematyki \\
  Wita Stwosza 57 \\
  80 -- 952 Gdańsk \\[1ex]
  \textit{e-mail: marta.kwela@mat.ug.edu.pl} 
}
\date{Bydgoszcz \today}
%---------------------
\begin{frame} 
  \titlepage 
\end{frame} 
%---------------------
%\section[Plan referatu]{}
%\frame{\tableofcontents}
%----------------------------------
%\section{Wstęp}
%----------------------------------
\begin{frame}											 
Dwa znaczenia pojęcia {\bf ideał}:
\uncover<1->{
$I \subseteq P(\real)$ rodzina zbiorów zamknięta na sumy
(jeśli przeliczane to $\sigma$-ideał) 
i branie podzbiorów. 	
}					
\uncover<2->{
	
}
	
	
\uncover<1->{
Topologia $\T$ $\Rightarrow$ ideał zbiorów nigdziegęstych
}
\end{frame}
%----------------------------------
\begin{frame}
\uncover<1->{
{Niezbędne definicje}
Oznaczenie: $\arithseq{a}{b} = \{an + b\colon n\in \cN_0\}$}
\uncover<2->{, oraz
$\SqrFr$ to wszystkie liczby wolne od kwadratów, 
czyli $\SqrFr = \{1,2,3,5,6,7,10,11,\ldots\mathrm{etc.}\}.$
}
\uncover<3->{
Rozważamy następujące topologie na $\cN$:

\begin{itemize}
\item \emph{Topologia Furstenberga} $\T_F$ \\
z bazą: $\B_F = \{\arithseq{a}{b} :\ b\leq a\}$,
\item \emph{Topologia Golomba} $\T_G$ \\
z bazą: $\B_G = \{\arithseq{a}{b} :\ (a,b)=1,\ b<a\}$,
\item \emph{Topologia Kircha} $\T_K$ \\
z bazą $\B_K = \{\arithseq{a}{b} :\ (a,b)=1,\ b<a,\ a\in\SqrFr\}$.
\end{itemize}
}
\end{frame}
%---------------------------------------------
\begin{frame}
\uncover<1->{
Dla każdej topologii na $\cN$
można rozważać ideał jej zbiorów nigdziegęstych, zatem mamy:
}
\uncover<2->{
\begin{block}{Definicja}
\begin{itemize}
\item \emph{Ideał Furstenberga} $\I_F$ zbiorów nigdzie gęstych w $\T_F$,
\item \emph{Ideał Golomba} $\I_G$ zbiorów nigdzie gęstych w $\T_G$,
\item \emph{Ideał Kircha} $\I_K$ zbiorów nigdzie gęstych w $\T_K$.
\end{itemize}
\end{block}
}
\end{frame}
%---------------------------------------------
\begin{frame}
Prosty przykład:
\uncover<2->{
\begin{block}{Przykład}
Zbiór $A = \{n! :\ n\in\N\}$ należy do wszystkich ideałów: $\I_F$, $\I_G$, $\I_K$.
\end{block}
}
\uncover<3->{
\begin{block}{Zwięzłe uzasadnienie}
Zauważmy że $A$ jest zbiorem domkniętym w topologii Kircha,
więc jest domknięty w każdej z pozostałych dwóch:
Gdy $x$ jest punktem skupienia $A$ w topologii Kircha 
to dla liczby pierwszej $p > x$ otoczenie
$\arithseq{p}{x}$ ,,haczy'' $A$ tylko w skończenie wielu punktach.
\uncover<4->{
  Dalej: $A$ jest brzegowy w każdej
z naszych $3$ topologiach, bo nie zawiera
ciągu arytmetycznego długości $3$.
}
\end{block}
}
\end{frame}
%---------------------------------------------
\begin{frame}
\uncover<1->{
\begin{block}{Przykład,P.Szczuka, artykuł(...)}
$\mathit{Primes}\in \I_F$, lecz $\mathit{Primes}$ 
są gęste w $\T_G$ i $\T_K$, zatem nie należą do
\end{block}
$\I_G$ jak i do $\I_K$).
}
\end{frame}
%---------------------------------------------
\begin{frame}
\uncover<1->{
Nasz cel: Jakie są relacje pomiędzy zdefiniowanymi ideałami,
jak są one ,,umocowane'' w obrębie ideałów $\cI_d$, $\cI_{\frac{1}{n}}$,
et cetera, jakie mają one własności...
}
\uncover<2->{
}
\end{frame}
%---------------------------------------------
\begin{frame}\frametitle{Wynik ,,pozytywny''}
\uncover<1->{
\begin{block}{Twierdzenie}
$\I_K \subseteq \I_G$.
\end{block} }
\uncover<2-> {
Szkic dowodu...}
\uncover<3-> {
Przypuśćmy że $X \not\in \I_G$...}
\uncover<4->{
Wówczas istnieje $\arithseq{a}{b}\in \B_G$ 
taki że dla każdego 
$\arithseq{c}{d}\subseteq \arithseq{a}{b}$, 
$\arithseq{c}{d}\in \T_G\setminus\{\emptyset\}$ mamy
$X\cap \arithseq{c}{d} \neq \emptyset$.}
%We need to show that there exists $\arithseq{a'}{b'}\in \B_K$ such that for every $\arithseq{c'}{d'}\subseteq \arithseq{a'}{b'}$ with $\arithseq{c'}{d'}\in \B_K$ we have $X\cap \arithseq{c'}{d'} \neq \emptyset$.
\uncover<5->{
Jak zdefiniujemy $a' := \prod_{p\in\Theta(a)}{p}$ 
(taką \`{a} la ,,bezkwadratową część'' liczby $a$)
i $b' := b \mod a'$. 
Wówczas, $\arithseq{a'}{b'} \in \B_K$.}
%%% bo $a' \in \SqrFr$, $(a',b') = 1$ (as $(a,b) = 1$), and $b'<a'$.
\uncover<6->{Niech teraz $\arithseq{c'}{d'} \subseteq \arithseq{a'}{b'}$ będzie takie że $\arithseq{c'}{d'}\in \B_K$ 
(wówczas $c' \in \SqrFr$, $(c',d')=1$, $d'<c'$, and $a'\mid c'$).}
%Note that $\arithseq{a}{b}\subseteq \arithseq{a'}{b'}$, $\arithseq{c'}{d'}\subseteq \arithseq{a'}{b'}$, 
%and observe that $\left(\frac{a}{a'}, \frac{c'}{a'}\right) = 1$ -- the only prime factors of $\frac{a}{a'}$ 
%are those that have already appeared in the factorization of $a'$, whereas all prime factors of $\frac{c'}{a'}$ 
%must be different from the prime factors of $a'$ since $c'$ is square-free. 
\uncover<7->{Kładziemy $A:= \arithseq{a}{b}\cap\arithseq{c'}{d'} \neq \emptyset$ i 
to będzie nasze szukane otoczenie.
}
\end{frame}
%---------------------------------------------
%---------------------------------------------
%---------------------------------------------
\begin{frame}
\uncover<1->{
Wszystkie omówione uprzednio relacje
pomiędzy naszymi ideałami daje się ująć
w postaci następującego diagramu:}
\uncover<2->{
\begin{center}
\begin{picture}(180,110)
\putrectangle{15}{15}{150}{75}
\put(20,94){\makebox(0,0){$\I_G$}}
\putrectangle{39}{30}{103}{48}
\put(42,82){\makebox(0,0){$\I_K$}}
\putrectangle{90}{8}{97}{76}
\put(192,88){\makebox(0,0){$\I_F$}}

\put(25,56){\makebox(0,0){?}}
\put(62,56){\makebox(0,0){$\arithseq{2}{2}$}}
\put(110,56){\makebox(0,0){$\{n!\}$}}
\put(152,56){\makebox(0,0){\ref{GFnotK}}}
\put(176,56){\makebox(0,0){$\SqrFr$}}
\end{picture}
\end{center}
}
\end{frame}
%---------------------------------------------
%---------------------------------------------
\begin{frame}
\uncover<1->{
Garść otwartych problemów:}

\begin{block}{Problem}
Czy prawdą jest że $\I_G\setminus (\I_K\cup \I_F) \not= \emptyset$?
\end{block}
\end{frame}
%---------------------------------------------
%---------------------------------------------
\begin{frame}{allowframebreaks}
\frametitle{Ten referat jest oparty m.in. na publikacjach:}
\beamertemplatebookbibitems
\begin{thebibliography}{10}{
\bibitem{MZ}
 {\sc Kwela, Marta, AN;} {Ideals of nowhere dense sets in some topologies on positive integers.} Topology Appl. 248 (2018), 149–163.
}
\bibitem{PS}
{\sc Szczuka P.}, {The connectedness of arithmetic progressions in Furstenberg's, Golomb's and Kirch's topologies},
Demonstratio Math. {\bf 43}(4) (2010) 899--909.
\end{thebibliography}
\end{frame}

\begin{frame}\frametitle{Ostatni slajd}
%\bigskip
\begin{center}{\Huge Dziękuję}\end{center}
\begin{center}{\Huge za}\end{center}
\begin{center}{\Huge Państwa Uwagę}\end{center}
%%%\begin{center}\includegraphics[height=2cm]{smile.png}\end{center}
\end{frame}

\end{document}

