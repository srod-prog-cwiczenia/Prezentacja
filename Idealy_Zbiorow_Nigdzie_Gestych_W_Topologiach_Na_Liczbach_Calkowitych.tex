\newcommand{\real}{\mathbb{R}}
\newcommand{\nnatural}{\mathbb{N}}
\newcommand{\cE}{\mathcal{E}}
\newcommand{\cF}{\mathcal{F}}
\newcommand{\cI}{\mathcal{I}}
\newcommand{\cM}{{\mathcal{M}}}
\newcommand{\cN}{{\mathcal{N}}}
\newcommand{\cW}{{\mathcal{W}}}
\newcommand{\cIMN}{\mathcal{I}_{\mathcal{N}, \mathcal{M}}}
\newcommand{\Fix}{\mathit{Fix}}

\newcommand{\N}{{\mathbb N}}
\newcommand{\Z}{{\mathbb Z}}
\newcommand{\R}{{\mathbb R}}
\newcommand{\Q}{{\mathbb Q}}
\newcommand{\Fin}{\textrm{Fin}}
\newcommand{\I}{\mathcal I}
\newcommand{\J}{\mathcal J}
\newcommand{\T}{\mathcal{T}}
\newcommand{\B}{\mathcal{B}}
\newcommand{\SqrFr}{\mathbb{SF}}
\newcommand{\calF}{\mathcal{F}}
\newcommand{\modulo}{\textrm{mod }}
\newcommand{\InfSubs}{[\N]^{\omega}}
\newcommand{\finbw}{\text{FinBW}}
\newcommand{\FinPart}{(\N)^{< \omega}}
\newcommand{\MB}{S^0}
\newcommand{\MBC}{\mathcal{MBC}}
\newcommand{\Seg}{\mathrm{Seg}}
\newcommand{\NULL}{\mathrm{NULL}}
\newcommand{\NWD}{\mathrm{NWD}}
\newcommand{\INULL}{\I_\mathrm{NULL}}			  
\newcommand{\cl}{\mathrm{cl}}
\newcommand{\interior}{\mathrm{int}}
\documentclass{beamer} 
%\documentclass[a4paper, 11pt, xcolor=dvipsnames]{beamer} 
\usepackage[polish]{babel}
\usepackage[utf8]{inputenc}
\usepackage{t1enc}
\usepackage{amsmath}
\usepackage{graphics} 
\usepackage{fancybox} 

%%%\newcommand{\arithseq}[2]{\{#1n + #2\}}
\newcommand{\arithseq}[2]{\langle#2, #1\rangle}
\newcommand{\Arithseq}[2]{\Big\langle#2, #1\Big\rangle}
%%% macros for the picture environment
\newcommand{\putrectangle}[4]{
  \multiput(#1,#2)(#3,0){2}{\line(0,1){#4}}
  \multiput(#1,#2)(0,#4){2}{\line(1,0){#3}}
}

%\usepackage{helvet}
%All font families are not available in every Beamer installation, but
%typically, at least some of the following families will be available:
%    serif          avant        bookman      chancery        charter
%     euler         helvet       mathtime       mathptm       mathptmx
%    newcent      palatino        pifont        utopia

\mode<presentation> {
\usetheme{Madrid} 
}
\usepackage{graphicx} % Allows including images
\usepackage{booktabs} % Allows the use of \toprule, \midrule and \bottomrule in tables
%%%%% pakiety spoza prezentacji - ,,spadly'' z macierzystego pliku
\usepackage{amssymb}
\usepackage{amsmath}
\usepackage{latexsym}

\usecolortheme[rgb={0.1,0.6,0.3}]{structure} 
%\usecolortheme[named=RubineRed]{structure} 
%\usecolortheme[named=CarnationPink]{structure}
%\usecolortheme[rgb={0.1,0.6,0.3}]{structure} 
\begin{document} 
%\setbeamercovered{transparent=15} %Ukryte (np przez uncover beda polwidoczne)
%\setbeamercovered{dynamic}        %Ukryte (np przez uncover beda polwidoczne, im blizej tym mocniej)
\setbeamercovered{transparent=0}
%%% to ponizej tez nie dziala...
%\transglitter[direction=90]
%%% to nizej mialo zmienic kolor bibliografii lecz niestety nie zadzialalo.
%\setbeamercolor{bibliography item}{fg=black}
%\setbeamercolor*{bibliography entry title}{fg=black}
%-----------------------------------
\title[Ideały zbiorów ND...]{
Ideały zbiorów nigdziegęstych w wybranych topologiach 
na liczbach naturalnych
} 
%\subtitle{}
\author{Marta Kwela \& AN} 
\institute[IM UG]{
  Universytet Gdański \\
  Institut Matematyki \\
  Wita Stwosza 57 \\
  80 -- 952 Gdańsk \\[1ex]
  \textit{e-mail: marta.kwela@mat.ug.edu.pl}\\
  \textit{e-mail: andrzej@mat.ug.edu.pl} 
}
\date{Bydgoszcz \today}
%---------------------
\begin{frame} 
  \titlepage 
\end{frame} 
%---------------------
%\section[Plan referatu]{}
%\frame{\tableofcontents}
%----------------------------------
%\section{Wstęp}
%----------------------------------
\begin{frame}\frametitle{Dwa oblicza pojęcia ideału}
\uncover<1->{
Dwa znaczenia pojęcia {\bf ideał}:
}

\uncover<2->{
$\I \subseteq P(\real)$ rodzina zbiorów zamknięta na sumy
(jeśli przeliczane to $\sigma$-ideał) 
i branie podzbiorów. (zamiast $\real$ może być $2^\omega$ 
albo jeszcze ogólnej: jakaś ustalona przestrzeń polska)
}	
\uncover<3->{
\begin{block}{Przykłady $\sigma$-ideałów:}
Zbiory miary zero, zbiory pierwszej kategorii, zbiory
sigma porowate, \textrm{etc.}
\end{block}
}				
\uncover<4->{
Topologia $\T$ $\Rightarrow$ $\I$ ideał zbiorów nigdziegęstych
}
\hrule
\uncover<5->{
$\I \subseteq P(\N)$ rodzina zbiorów zamknięta
na sumy i branie podzbiorów. 
}
\uncover<6->{
\begin{block}{Przykłady klasycznych ideałów}
Ideał $\I_d = \{A\subseteq\N\colon \lim_{n\to\infty} \frac{|A \cap
\{1,\ldots,n\}|}{n} = 0$

Ideał $\I_{\frac{1}{n}} = \{A\subseteq\N\colon \sum_{n\in A} \frac{1}{n} <
\infty\}$
\end{block}
}
\uncover<7->{
Mamy: $\I_{\frac{1}{n}} \subseteq \I_d$.
}

\uncover<8->{
??? (jaka topologia?) $\Rightarrow$ $\cI$ ideał zbiorów nigdziegęstych w tej top.
}
\end{frame}
%----------------------------------
\begin{frame}\frametitle{Garść oznaczeń i definicji}
\uncover<1->{
\begin{block}{Niezbędne definicje}
Oznaczenie: $\arithseq{a}{b} = \{an + b\colon n\in \cN_0\}$
}

\uncover<2->{
$\SqrFr$ to wszystkie liczby wolne od kwadratów, 
czyli $\SqrFr = \{1,2,3,5,6,7,10,11,\ldots\mathrm{etc.}\}.$
\end{block}}
\uncover<3->{
Rozważamy następujące topologie na $\cN$:

\begin{itemize}
\item \emph{Topologia Furstenberga} $\T_F$ \\
z bazą: $\B_F = \{\arithseq{a}{b} :\ b\leq a\}$,
\item \emph{Topologia Golomba} $\T_G$ \\
z bazą: $\B_G = \{\arithseq{a}{b} :\ (a,b)=1,\ b<a\}$,
\item \emph{Topologia Kircha} $\T_K$ \\
z bazą $\B_K = \{\arithseq{a}{b} :\ (a,b)=1,\ b<a,\ a\in\SqrFr\}$.
\end{itemize}
}
\end{frame}
%---------------------------------------------
\begin{frame}\frametitle{Główne postacie akcji}
\uncover<1->{
Dla każdej topologii na $\N$
można rozważać ideał jej zbiorów nigdziegęstych, zatem mamy:
}
\uncover<2->{
\begin{block}{Definicja}
\begin{itemize}
\item<2-> \emph{Ideał Furstenberga} $\I_F$ zbiorów nigdziegęstych w $\T_F$,
\item<3-> \emph{Ideał Golomba} $\I_G$ zbiorów nigdziegęstych w $\T_G$,
\item<4-> \emph{Ideał Kircha} $\I_K$ zbiorów nigdziegęstych w $\T_K$.
\end{itemize}
\end{block}
}
\end{frame}
%---------------------------------------------
\begin{frame}\frametitle{Prosty przykład}
Prosty przykład:
\uncover<2->{
\begin{block}{Przykład}
Zbiór $A = \{n! :\ n\in\N\}$ należy do wszystkich ideałów: $\I_F$, $\I_G$, $\I_K$.
\end{block}
}
\uncover<3->{
\begin{block}{Zwięzłe uzasadnienie}
\begin{itemize}
\item<3->Zauważmy że $A$ jest zbiorem domkniętym w topologii Kircha,
więc jest domknięty w każdej z pozostałych dwóch:
Gdy $x$ jest punktem skupienia $A$ w topologii Kircha 
to dla liczby pierwszej $p > x$ otoczenie
$\arithseq{p}{x}$ ,,haczy'' $A$ tylko w skończenie wielu punktach.
\item<4->Dalej: $A$ jest brzegowy w każdej
z naszych $3$ topologiach, bo nie zawiera
ciągu arytmetycznego długości $3$.
\end{itemize}
\end{block}}
\end{frame}
%---------------------------------------------
\begin{frame}[label=powrotPS]
\uncover<1->{
\begin{block}{Przykład,P.Szczuka, \hyperlink{bibliografia}{\beamergotobutton{publikacja}} }
$\mathit{Primes}\in \I_F$, lecz $\mathit{Primes}$ 
są gęste w $\T_G$ i $\T_K$, zatem nie należą do $\I_G$ jak i do $\I_K$).
\end{block}
}
\end{frame}
%---------------------------------------------
\begin{frame}
\uncover<1->{
\begin{block}{Nasz cel:}
Jakie są relacje pomiędzy zdefiniowanymi ideałami,
jak są one ,,umocowane'' w obrębie ideałów $\cI_d$, $\cI_{\frac{1}{n}}$,
et cetera, jakie mają one własności...
\end{block}
}
\end{frame}
%---------------------------------------------
\begin{frame}\frametitle{Wynik ,,pozytywny''}
\uncover<1->{
\begin{block}{Twierdzenie}
$\I_K \subseteq \I_G$.
\end{block} }
\uncover<2-> {
Szkic dowodu...}
\uncover<3-> {
Przypuśćmy że $X \not\in \I_G$...}
\uncover<4->{
Wówczas istnieje $\arithseq{a}{b}\in \B_G$ 
taki że dla każdego 
$\arithseq{c}{d}\subseteq \arithseq{a}{b}$, 
$\arithseq{c}{d}\in \T_G\setminus\{\emptyset\}$ mamy
$X\cap \arithseq{c}{d} \neq \emptyset$.}
%We need to show that there exists $\arithseq{a'}{b'}\in \B_K$ such that for every $\arithseq{c'}{d'}\subseteq \arithseq{a'}{b'}$ with $\arithseq{c'}{d'}\in \B_K$ we have $X\cap \arithseq{c'}{d'} \neq \emptyset$.
\uncover<5->{
Gdy zdefiniujemy $a' := \prod_{p\in\Theta(a)}{p}$ 
(taką \`{a} la ,,bezkwadratową część'' liczby $a$)
i $b' := b \mod a'$. 
Wówczas, $\arithseq{a'}{b'} \in \B_K$.}
%%% bo $a' \in \SqrFr$, $(a',b') = 1$ (as $(a,b) = 1$), and $b'<a'$.
\uncover<6->{Niech teraz $\arithseq{c'}{d'} \subseteq \arithseq{a'}{b'}$ będzie takie że $\arithseq{c'}{d'}\in \B_K$ 
(wówczas $c' \in \SqrFr$, $(c',d')=1$, $d'<c'$, and $a'\mid c'$).}
%Note that $\arithseq{a}{b}\subseteq \arithseq{a'}{b'}$, $\arithseq{c'}{d'}\subseteq \arithseq{a'}{b'}$, 
%and observe that $\left(\frac{a}{a'}, \frac{c'}{a'}\right) = 1$ -- the only prime factors of $\frac{a}{a'}$ 
%are those that have already appeared in the factorization of $a'$, whereas all prime factors of $\frac{c'}{a'}$ 
%must be different from the prime factors of $a'$ since $c'$ is square-free. 
\uncover<7->{Kładziemy $A:= \arithseq{a}{b}\cap\arithseq{c'}{d'} \neq \emptyset$ i 
to będzie nasze szukane otoczenie.
}
\end{frame}
%---------------------------------------------
\begin{frame}[label=GFnotK]
\uncover<1->{
\begin{block}{Twierdzenie}
$(\I_G\cap \I_F) \backslash \I_K \not= \emptyset$.
\end{block}}
\uncover<2->{
\begin{alertblock}{Konstrukcja jest nieco zawiła}\end{alertblock}
Niech $\mathcal{C} := \{\arithseq{a}{b}\in \B_K :\ \arithseq{a}{b}\subseteq \arithseq{2}{1}\}$. i  $\{C_k :\ k\in\N\}$ numeracja elementów $\mathcal{C}$.

Jak konstruujemy $X \in (\I_G\cap \I_F) \setminus \I_K$?
Dla każdego $k\in\N$ wybieramy $x_k$ tak aby:
\begin{itemize}
	\item $x_k\in C_k$,
	\item $x_k\in \arithseq{2^k}{1}$.
\end{itemize}

Ta konstrukcja jest możliwa gdyż $C_k \cap \arithseq{2^k}{1}$ 
jest zawsze niepusty 

(if $C_k = \arithseq{a_k}{b_k}$, then $a_k$ is even and square-free, and $\left(\frac{a_k}{2},\frac{2^k}{2}\right)=\left(\frac{a_k}{2},2^{k-1}\right)=1$ as $\frac{a_k}{2}$ is odd; both $\arithseq{a_k}{b_k}$ and $\arithseq{2^k}{1}$ are subsequences of $\arithseq{2}{1}$, so, by Lemma \ref{lemCRT}, their intersection is nonempty). 

Niech $X := \{x_k :\ k\in\N\}$.

Można sprawdzić że $X \notin \I_K$. 

%Indeed, the set $\arithseq{2}{1}\in\B_K$ has a property that for every $\arithseq{c}{d}\subseteq \arithseq{2}{1}$ with $\arithseq{c}{d}\in \B_K$ (so $\arithseq{c}{d}=C_{k_0}$ for some $k_0\in\N$) we have $X\cap \arithseq{c}{d} \neq \emptyset$ (as it contains $x_{k_0}$).
  Wybierzmy teraz dowolny $\arithseq{a}{b} \in \B_F$. 
Wystarczy pokazać że istnieje niepusty
$V \subseteq \arithseq{a}{b}$, $X\cap V = \emptyset$ taki że
$V\in \T_F$ i gdy $\arithseq{a}{b}\in \B_G$ to $V\in\T_G$.

  Na początek załóżmy że $2\not | a$. Niech 
$V = (\arithseq{a}{b} \cap \arithseq{4}{3}) \setminus \{x_1\}$. 
Then $V \not= \emptyset$ because $(a,4) = 1$ so 
$V\in \T_F$ and, if $\arithseq{a}{b}\in \B_G$, then $V\in \T_G$.
If $k > 1$, then $x_k \in \arithseq{4}{1}$ so $V\cap X = \emptyset$.
  
  Załóżmy teraz że $2|a$ i wybierzmy największe $m\in \N$
takie, że $2^m|a$. Gdy $b$ jest parzyste, to $\arithseq{a}{b}\cap X = \emptyset$
i kładziemy $V = \arithseq{a}{b}$. Załóżmy zatem że 
$b$ jest nieparzyste. If $b\not\equiv 1(\modulo 2^m)$, let
$V = \arithseq{a}{b}\setminus \{x_1,\ldots, x_{m-1}\}$.
Then $\arithseq{a}{b}\cap \arithseq{2^m}{1} = \emptyset$ so $V\cap X = \emptyset$.

  Thus we may assume that $b\equiv 1(\modulo 2^m)$. Let 
\[
V = (\arithseq{a}{b} \cap \arithseq{2^{m+1}}{1 + 2^m}) \setminus \{x_1,\ldots, x_{m}\}.
\]
Then $V \not= \emptyset$ because $(\frac{a}{2^m} , \frac{2^{m+1}}{2^m}) = 1$
and $\arithseq{a}{b}$ and $\arithseq{2^{m+1}}{1 + 2^m}$
are contained in $\arithseq{2^m}{1}$. 
If $k > m$, then $x_k \in \arithseq{2^{m+1}}{1}$ so 
$V\cap X = \emptyset$. Also $V\in\T_F$ and, if $\arithseq{a}{b} \in \B_G$, then
$V \in \T_G$.
}
\hyperlink{diagramik}{\beamerreturnbutton{Do diagramu}}
\end{frame}
%---------------------------------------------
%---------------------------------------------
\begin{frame}[label=diagramik]
\uncover<1->{
Wszystkie omówione uprzednio relacje
pomiędzy naszymi ideałami daje się ująć
w postaci następującego diagramu:}
\uncover<2->{
\begin{center}
\begin{picture}(180,110)
\putrectangle{15}{15}{150}{75}
\put(20,94){\makebox(0,0){$\I_G$}}
\putrectangle{39}{30}{103}{48}
\put(42,82){\makebox(0,0){$\I_K$}}
\putrectangle{90}{8}{97}{76}
\put(192,88){\makebox(0,0){$\I_F$}}

\put(25,56){\makebox(0,0){?}}
\put(62,56){\makebox(0,0){$\arithseq{2}{2}$}}
\put(110,56){\makebox(0,0){$\{n!\}$}}
\put(152,56){\makebox(0,0){\hyperlink{GFnotK}{\beamerreturnbutton{Ex}}}}
\put(176,56){\makebox(0,0){$\SqrFr$}}
\end{picture}
\end{center}
}
\end{frame}
%---------------------------------------------
%---------------------------------------------
\begin{frame}\frametitle{Szczypta otwartych problemów}
\uncover<1->{Otwarte problemy:}

\begin{block}{Problem}
\begin{enumerate}
\item<1>Czy prawdą jest że $\I_G\setminus (\I_K\cup \I_F) \not= \emptyset$?
\end{enumerate}
\end{block}
\end{frame}
%---------------------------------------------
%---------------------------------------------
\begin{frame}[label=powrotIF]\frametitle{Dalsze ścieżki}
\uncover<1->{
Jakie własności mają nasze ideały i jak są umocowane 
wśród kolekcji klasycznych ideałów?
}
\uncover<2->{
\begin{block}{Własności ideałów}
Ideał może być: P ideałem, analitycznym, tall, mieć własność BW,
FinBW, być typu $F_{\sigma}$, \textrm{etc.} 
\end{block}
}
\uncover<3->{
\begin{block}{Miriady klasycznych ideałów, np.}
\begin{itemize}
\item<4->
$\cW$ - ideał van der Waerdena = 
$\{A\subseteq \N\colon\textrm{A nie zawiera ciągów arytmetycznych dowolnej
długości}\}$
\item<5->
$\NWD(\Q):=\left\{A\subseteq\mathbb{Q}:\ \cl_\R(A) \textrm{ jest nigdziegęsty }\right\},$
\item<6->
$\NULL(\Q):=\left\{A\subseteq\mathbb{Q}:\ \cl_\R(A) \textrm{ jest miary Lebesgue'a zero}\right\}$.
W \hyperlink{bibliografia}{\beamergotobutton{pracy}} I. Farah and S. Solecki udowodnili że
$\NWD(\Q)$ i $\NULL(\Q)$ nie są izomorficzne.
\end{itemize}
\end{block}
}
\end{frame}
%---------------------------------------------
\begin{frame}\frametitle{Cechy ideałów zbiorów ND}
\uncover<1->{
\begin{block}{Ważka obserwacja:}
Wszystkie rozważane wcześniej topologie na $\N$ 
(Furstenberga, Golomba, Kircha, etc.) mają
bazę przeliczalną (innymi słowy: ,,spełniają drugi aksjomat przeliczalności'').
\end{block}
}
\uncover<2->{
\begin{block}{Notka na marginesie:}
Nie każda topologia na zbiorze przeliczalnym musi spełniać
nawet pierwszy aksjomat przeliczalności.
kontrprzykład: Tak zwana {\color{red}przestrzeń Arensa–Forta}.
\end{block}
}
\end{frame}
%---------------------------------------------
\begin{frame}\frametitle{Chwilowo rozważamy ideały w pierwszym sensie}
\uncover<1->{
\begin{block}{Idea Marczewskiego i Burstina}
Dla dowolnej rodziny $\cF \subseteq P(X)\setminus\{\emptyset\}$
definiujemy:
\begin{enumerate}
\item<2-> 
$S(\cF) = \{A\subseteq X\colon \forall_{T\in\cF}\exists_{W\in\cF} 
W \subseteq T \cap A \vee W \subseteq T \cap A^c\}$
\item<3-> 
$S_0(\cF) = \{A\subseteq X\colon \forall_{T\in\cF}\exists_{W\in\cF} W \subseteq T \cap A^c\}$
\end{enumerate}
\end{block}
}
\uncover<4->{
\begin{block}{Prosta obserwacja}
Rodzina $S(\cF)$ jest ciałem zbiorów, zaś $S_0(\cF)$ jest ideałem.
\end{block}
}
\end{frame}
%---------------------------------------------
\begin{frame}\frametitle{Wracamy do ideałów na $\cN$}
\uncover<1->{
\begin{block}{Definicja}
Ideał $\cI\subseteq P(\N)$ nazywamy ideałem 
\emph{MB-przeliczalnie reprezentowalnym}(skrót: $\MBC$) gdy 
istnieje \emph{przeliczalna} rodzina $\cF \subseteq P(\N)$
zbiorów nieskończonych taka, że 
$\cI = S^0(\cF) = 
\left\{A\subseteq \N :\ \forall_{F\in\mathcal{F}}\ \exists_{G\in\mathcal{F},\ G\subseteq F}\ A\cap G=\emptyset\right\}$
\end{block}
}

\begin{block}{Uwaga}
Ideały $\I_F$, $\I_G$ i $\I_K$ są $\MBC$.
\end{block}

\end{frame}
%---------------------------------------------
\begin{frame}\frametitle{Nasze ideały są {\color{red}tall}}
\uncover<1->{
\begin{block}{Definicja}
Mówimy że ideał $\mathcal{I}\subseteq\mathcal{P}(\N)$ jest 
{\color{red}tall} gdy każdy nieskończony podzbiór $\N$ zawiera nieskończony podzbiór
z ideału $\mathcal{I}$.
\end{block}}
\uncover<2->{
\begin{block}{Twierdzenie}
Niech $\mathcal{F}\subseteq P(\N)$ będzie przeliczaną rodziną złożoną
ze zbiorów nieskończonych. Załóżmy ponadto że:
\begin{itemize}
\item<3-> $\mathcal{F}$ ma własność \emph{base-like},
czyli 
$\forall_{F_1, F_2\in\mathcal{F},\ F_1\cap F_2\neq\emptyset}\ \exists_{H\in\mathcal{F}}
\ H\subseteq F_1\cap F_2$
\item<4-> $\mathcal{F}$ ma własność \emph{splittingu}, 
czyli
$\forall_{F\in\mathcal{F}}\ \exists_{F_1,F_2\in\mathcal{F},\ F_1\cup F_2
 \subseteq F}\ F_1\cap F_2 = \emptyset$
\end{itemize}
\uncover<5->{Wówczas ideał $\I=S^0(\mathcal{F})$ jest {\color{red}tall}.}
\end{block}
}
\uncover<5->{
\begin{block}{Wniosek}
Ideały $\I_F$, $\I_G$ oraz $\I_K$ są wszystkie {\color{red}tall}.
\end{block}
}
\end{frame}
%---------------------------------------------
\begin{frame}\frametitle{Wyznaczenie klasy borelowskiej ideału}
\uncover<1->{
\begin{block}{Ważkie pytanie:}
Każdy ideał $\I$ na $\N$ może być
traktowany jako pewien specyficzny podzbiór 
zbioru Cantora $2^{\omega}$, mianowicie
$\I \subseteq P(\N) \simeq 2^\omega$.
W ten sposób możemy pytać o klasę borelowską ideału.
\end{block}
}
\uncover<2->{
\begin{block}{Twierdzenie}
Niech $\mathcal{F}\subseteq P(\N)$ będzie przeliczaną rodziną złożoną
ze zbiorów nieskończonych. Załóżmy ponadto że:
\begin{itemize}
\item<3-> $\mathcal{F}$ ma własność \emph{base-like},
czyli 
$\forall_{F_1, F_2\in\mathcal{F},\ F_1\cap F_2\neq\emptyset}\ \exists_{H\in\mathcal{F}}
\ H\subseteq F_1\cap F_2$
\item<4-> $\mathcal{F}$ ma własność \emph{splittingu}, 
czyli
$\forall_{F\in\mathcal{F}}\ \exists_{F_1,F_2\in\mathcal{F},\ F_1\cup F_2
 \subseteq F}\ F_1\cap F_2 = \emptyset$
\end{itemize}
\uncover<5->{Wówczas ideał $\I=S^0(\mathcal{F})$ {\color{red}nie} jest typu $F_{\sigma}$.}
\end{block}
}
\uncover<6->{
...łącząc to z faktem iż każdy ideał $\MBC$ jest
typu $F_{\sigma\delta}$ dostajemy:
\begin{block}{Twierdzenie}
$\I_F$, $\I_G$ oraz $\I_K$ są ideałami typu $F_{\sigma\delta}$ lecz już nie $F_{\sigma}$.
\end{block}}
\end{frame}
%---------------------------------------------
\begin{frame}\frametitle{Topologiczna własność ideału Furstenberga}
\uncover<2->{
\begin{block}{Twierdzenie}
Ideały $\I_F$ oraz $\NWD(\Q)$ są izomorficzne.
\end{block}
}
\uncover<3->{
\begin{block}{Szkic dowodu}
Wystarczy zauważyć że $\N$ z topologią Furstenberga jest
homeomorficzna z przestrzenią $\Q$ (z naturalną topologią), 
wynika to z klasycznego twierdzenia Sierpińskiego z roku 1920 
charakteryzującego liczby wymierna jako jedyną przeliczalną
przestrzeń metryzowalną bez punktów izolowanych.
\end{block}
}
\end{frame}

%---------------------------------------------
%---------------------------------------------
%---------------------------------------------
%---------------------------------------------
%---------------------------------------------
%---------------------------------------------
%---------------------------------------------
%---------------------------------------------
%---------------------------------------------
%---------------------------------------------
\begin{frame}[label=bibliografia]{allowframebreaks}
\frametitle{Ten referat jest oparty m.in. na publikacjach:}
\beamertemplatebookbibitems
\begin{thebibliography}{10}{
\bibitem{FS}
{\sc Farah I., Solecki S.}, {Two $F_{\sigma\delta}$ ideals},
Proc. Amer. Math. Soc. {\bf 131}(6) (2003) 1971--1975.

\hyperlink{powrotIF}{\beamerreturnbutton{Powrót}}

\bibitem{MZ}
 {\sc Kwela, Marta, AN;} {Ideals of nowhere dense sets in some topologies on positive integers.} Topology Appl. 248 (2018), 149–163.
}
\bibitem{PS}
{\sc Szczuka P.}, {The connectedness of arithmetic progressions in Furstenberg's, Golomb's and Kirch's topologies},
Demonstratio Math. {\bf 43}(4) (2010) 899--909.

\hyperlink{powrotPS}{\beamerreturnbutton{Powrót}}

\end{thebibliography}
\end{frame}

\begin{frame}\frametitle{Ostatni slajd}
%\bigskip
\begin{center}{\Huge Dziękuję}\end{center}
\begin{center}{\Huge za}\end{center}
\begin{center}{\Huge Państwa Uwagę}\end{center}
%%%\begin{center}\includegraphics[height=2cm]{smile.png}\end{center}
\end{frame}

\end{document}

%%%%%%%% szablonik na kolejny slajd:
\begin{frame}\frametitle{XXX}
\uncover<1->{
XXX
}
\uncover<2->{
\begin{block}{XXX}
XXX
\end{block}
}
\end{frame}

