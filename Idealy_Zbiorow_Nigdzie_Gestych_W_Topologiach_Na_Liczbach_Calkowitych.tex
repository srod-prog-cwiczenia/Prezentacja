\newcommand{\real}{\mathbb{R}}
\newcommand{\nnatural}{\mathbb{N}}
\newcommand{\cE}{\mathcal{E}}
\newcommand{\cF}{\mathcal{F}}
\newcommand{\cI}{\mathcal{I}}
\newcommand{\cM}{{\mathcal{M}}}
\newcommand{\cN}{{\mathcal{N}}}
\newcommand{\cIMN}{\mathcal{I}_{\mathcal{N}, \mathcal{M}}}
\newcommand{\Fix}{\mathit{Fix}}

\documentclass{beamer} 
%\documentclass[a4paper, 11pt, xcolor=dvipsnames]{beamer} 
\usepackage[polish]{babel}
\usepackage[utf8]{inputenc}
\usepackage{t1enc}
\usepackage{amsmath}
\usepackage{graphics} 
\usepackage{fancybox} 

%\usepackage{helvet}
%All font families are not available in every Beamer installation, but
%typically, at least some of the following families will be available:
%    serif          avant        bookman      chancery        charter
%     euler         helvet       mathtime       mathptm       mathptmx
%    newcent      palatino        pifont        utopia

\mode<presentation> {
\usetheme{Madrid} 
}
\usepackage{graphicx} % Allows including images
\usepackage{booktabs} % Allows the use of \toprule, \midrule and \bottomrule in tables

\usecolortheme[rgb={0.1,0.6,0.3}]{structure} 
%\usecolortheme[named=RubineRed]{structure} 
%\usecolortheme[named=CarnationPink]{structure}
%\usecolortheme[rgb={0.1,0.6,0.3}]{structure} 
\begin{document} 
%\setbeamercovered{transparent=15} %Ukryte (np przez uncover beda polwidoczne)
\setbeamercovered{dynamic}        %Ukryte (np przez uncover beda polwidoczne, im blizej tym mocniej)
%%% to ponizej tez nie dziala...
%\transglitter[direction=90]
%%% to nizej mialo zmienic kolor bibliografii lecz niestety nie zadzialalo.
%\setbeamercolor{bibliography item}{fg=black}
%\setbeamercolor*{bibliography entry title}{fg=black}
%-----------------------------------
\title[On special kind of ideals...]{
On special kind of ideals of the real line generated by partitions into measure null and first category sets.
} 
%\subtitle{}
\author{Andrzej Nowik} 
\institute[IM UG]{
  University of Gdańsk \\
  Institute of Mathematics \\
  Wita Stwosza 57 \\
  80 -- 952 Gdańsk \\
  Poland\\[1ex]
  \textit{e-mail: andrzej@mat.ug.edu.pl} 
}
\date{Frontiers of Selection Principles, 
Warsaw \today}
%---------------------
\begin{frame} 
  \titlepage 
\end{frame} 
%---------------------
%\section[Outline of the talk]{}
%\frame{\tableofcontents}
%----------------------------------
%\section{Introduction}
\begin{frame}
\uncover<1->{
\begin{block}{The starting point}
That is the starting point for the presentation:

  A partition of the real line (vs. the Cantor set $2^\omega$)
into the set of measure null and first category:
  $\real = M \cup N$, where $M \in \cM$ and $N \in \cN$, where
$\cM$ is the collection of all first category sets and
$\cN$ is the collection of all sets of measure zero.
We can find such a partition with the property that 
$M$ is an $F_{\sigma}$ set and $N$ is a $G_{\delta}$ set.

\end{block}
}
\uncover<2->{
\begin{block}{Definition} 
  \begin{itemize}
  \item<2->
For such partition we have that $M + M$ and $N + N$ have nonempty
interior (it follows from the Steinhaus Theorem).
  \end{itemize}
\end{block}
}
\uncover<3->{
\begin{block}{Example}
It is known that there exists a partition of the real line into $G_{\sigma}$ set
$N$ of measure zero and $F_{\sigma}$ set $M$ of first category
such that $\mathit{int}(M + N) = \emptyset$.
\end{block}
}
\end{frame}
%---------------------------------------------
\begin{frame}
\uncover<1->{
\begin{block}{An easy observation}
If $M,N$ is such partition as above then $(M + N)^c$ is a set of
measure null and first category.
\end{block}
}
\uncover<2->{
\begin{block}{The ideal}
So, it is natural to define (and ask about its properties) 
the $\sigma$-ideal $\cIMN$ as the $\sigma$-ideal
generated by the family of sets 
$(M + N)^c$, where $\{N,M\}$ are all partitions
of $\real$ into sets $G_{\delta}$ of measure null and
$F_{\sigma}$ of first category, respectively.
\end{block}
}
\end{frame}
%---------------------------------------------
%%% tutaj potencjalna ramka z twierdzeniem o Fix z pliku imn.
%---------------------------------------------
\begin{frame}{allowframebreaks}
\frametitle{This talk is based on}
\beamertemplatebookbibitems
\begin{thebibliography}{10}{
\bibitem{MZ}
  {\sc M.Michalski, Sz.Żeberski}, 
  {\em Some properties of ${\mathcal I}-Luzin sets,$}
  Topology and its Applications 55 (2015) Vol 198, 122-135.
}
\end{thebibliography}
\end{frame}

\begin{frame}\frametitle{The last slide}
%\bigskip
\begin{center}{\Huge Thank You}\end{center}
\begin{center}{\Huge for}\end{center}
\begin{center}{\Huge Your Attention}\end{center}
%%%\begin{center}\includegraphics[height=2cm]{smile.png}\end{center}
\end{frame}

\end{document}

